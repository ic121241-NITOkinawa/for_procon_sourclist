\documentclass[11pt,a4j]{jarticle}
\usepackage[dvipdfmx]{graphicx}
\usepackage{amsmath}
\usepackage{slashbox}
\usepackage{subfigure} 
\usepackage{here}
\usepackage[T1]{fontenc}
\usepackage{url}
\usepackage{verbatim}
%\usepackage{seqsplit}
\usepackage{listings, jlisting}


\setlength{\textwidth}{170mm}
\setlength{\evensidemargin}{-5mm}
\setlength{\oddsidemargin}{-5mm}


\makeatletter
\newcommand{\figcaption}[1]{\def\@captype{figure}\caption{#1}}
\newcommand{\tblcaption}[1]{\def\@captype{table}\caption{#1}}
\makeatother

\begin{document}
\vspace*{-20mm}
{\Large

\begin{flushleft}
第26回プログラミングコンテスト
\end{flushleft}

\vspace{-15mm}
\begin{flushright}
プログラムソースリスト
\end{flushright}
}

\vspace{20mm}

\begin{center}
{\huge

{自由部門 : 発表順番号(登録番号)  4(20041)}

\vspace{20mm}
タイトル:「SwipeTalk」


\vspace{20mm}
学校名:沖縄工業高等専門学校\\

\hspace{0mm}
\vspace{0mm}

\begin{tabular}{r l l}
氏名:& 西原 & 希咲\\
	 & 辺土名 & 朝飛 \\
	 & 与座 & 章宙 \\
	 & 當間 & 環 \\
指導教員:& 正木 & 忠勝 \\
\end{tabular}
}

\end{center}


\thispagestyle{empty}
\newpage
\setcounter{page}{1}



\tableofcontents


\thispagestyle{empty}
\newpage
\setcounter{page}{1}




\section{SwipeTalk}
\subsection{DocumentActivity.java}
	\verbatiminput{./srcSwipeTalk/DocumentActivity.java}
	\newpage
\subsection{HueAdapter.java}
	\verbatiminput{./srcSwipeTalk/HueAdapter.java}
	\newpage	
\subsection{ImageCaptureActivity.java}
	\verbatiminput{./srcSwipeTalk/ImageCaptureActivity.java}
	\newpage
\subsection{MainActivity.java}
	\verbatiminput{./srcSwipeTalk/MainActivity.java}
	\newpage
\subsection{MessageActivity.java}
	\verbatiminput{./srcSwipeTalk/MessageActivity.java}
	\newpage
\subsection{MultiSelectableGridView.java}
	\verbatiminput{./srcSwipeTalk/MultiSelectableGridView.java}
	\newpage
\subsection{Phone\_PositionEstimation.java}
%	\verbatiminput{./srcSwipeTalk/Phone_PositionEstimation.java}
%	\newpage
\begin{verbatim}
\verb/\capwidth = 65mm

package com.example.mi111331.ibeacon;

import android.hardware.Sensor;
import android.hardware.SensorEvent;
import android.hardware.SensorEventListener;
import android.hardware.SensorManager;
import android.os.Handler;
import android.os.RemoteException;
import android.support.v7.app.ActionBarActivity;
import android.os.Bundle;
import android.util.Log;
import android.view.Menu;
import android.view.MenuItem;
import android.view.View;
import android.widget.ArrayAdapter;
import android.widget.Button;
import android.widget.ListView;

import org.altbeacon.beacon.Beacon;
import org.altbeacon.beacon.BeaconConsumer;
import org.altbeacon.beacon.BeaconManager;
import org.altbeacon.beacon.BeaconParser;
import org.altbeacon.beacon.Identifier;
import org.altbeacon.beacon.MonitorNotifier;
import org.altbeacon.beacon.RangeNotifier;
import org.altbeacon.beacon.Region;

import java.util.ArrayList;
import java.util.Collection;
import java.util.List;


class MainActivity extends ActionBarActivity implements BeaconConsumer {
    // BeaconConsumerインターフェースを実装
    private static String TAG = "AltBeacon Sample";
    // BeaconManagerクラスの変数を定義
    private BeaconManager beaconManager;

    //スタートボタンがおされたか判定
    int flag_start = 0;
    //4つのビーコンすべてのRSSIがそろったか判定
    static boolean all_rssi_flag = false;

    //4つのビーコンのRSSIを格納 スマホの位置特定に使用
    static int[] rssi = {0, 0, 0, 0};

    //スマホの座標比を格納する
    static double[] phone_ratio = {0, 0};

    //ビーコンまでの距離
    static double[] distance = {-200, -200, -200, -200};

    //スマホの方位角を格納する変数 北から現在向いている方向の角度
    static double phone_slope = 0;

    // iBeaconのデータを認識するためのParserフォーマット
    public static final String IBEACON_FORMAT = "m:2-3=0215,i:4-19,i:20-21,i:22-23,p:24-24";
    // UUIDの作成
    /*リストビュー*/
    static ArrayAdapter<String> DataList;
    static ListView listView;

    Handler mHandler = new Handler();

    /*スマホの方角測定時に使用*/
    //センサーマネージャー
    private SensorManager manager;
    //センサーイベントリスナー
    private SensorEventListener listener;

    //加速度の値
    private float[] fAccell = null;
    //地磁気の値
    private float[] fMagnetic = null;

    private static final int MATRIX_SIZE = 16;
    float[] inR = new float[MATRIX_SIZE];
    float[] outR = new float[MATRIX_SIZE];
    float[] I = new float[MATRIX_SIZE];


    @Override
    protected void onCreate(Bundle savedInstanceState) {
        super.onCreate(savedInstanceState);
        setContentView(R.layout.activity_main);

        // staticメソッドでBeaconManagerクラスのインスタンスを取得
        beaconManager = BeaconManager.getInstanceForApplication(this);
        // BeaconParseをBeaconManagerに設定
        beaconManager.getBeaconParsers().add(new BeaconParser().setBeaconLayout(IBEACON_FORMAT));
        try {
            //beaconスキャン間隔を100msecに指定
            beaconManager.setForegroundScanPeriod(100l);
            //beaconスキャン間隔を100msecに指定
            beaconManager.setForegroundBetweenScanPeriod(0l);

            beaconManager.updateScanPeriods();
        } catch (RemoteException e) {
            e.printStackTrace();
        }

        Log.i(TAG, "00_StartButton");
        Log.i(TAG, "01_list");
        //ibeaconから取得したデータをAPP上に表示
        listView = (ListView)findViewById(R.id.data_list);
        List <String> data = new ArrayList<String>();


        /*スマホの方角特定のセンサー*/
        //センサーマネージャーを取得
        manager = (SensorManager)getSystemService(SENSOR_SERVICE);
        //センサーのイベントリスナーを登録
        listener = new SensorEventListener() {
            //値変更時b
            @Override
            public void onSensorChanged(SensorEvent sensorEvent) {
                //センサーの種類で値を取得
                switch (sensorEvent.sensor.getType()) {
                    //加速度
                    case Sensor.TYPE_ACCELEROMETER:
                        fAccell = sensorEvent.values.clone();
                        break;

                    //地磁気
                    case Sensor.TYPE_MAGNETIC_FIELD:
                        fMagnetic = sensorEvent.values.clone();
                        break;
                }
                //両方確保した時点でチェック
                if(fAccell != null && fMagnetic != null){
                    //回転行列を得る
                    SensorManager.getRotationMatrix(inR, I,fAccell, fMagnetic);

                    //ワールド座標とデバイス座標のマッピングを変換する
                    SensorManager.remapCoordinateSystem(inR, SensorManager.AXIS_X, SensorManager.AXIS_Y, outR);

                    //姿勢を得る
                    float[] fAttitube = new float[3];
                    SensorManager.getOrientation(outR, fAttitube);

                    for(int i = 0; i < 3; i++){
                        fAttitube[i] = (float)(fAttitube[i] * 180 / Math.PI);
                        fAttitube[i] = (fAttitube[i] < 0) ? fAttitube[i] + 360 :  fAttitube[i];
                    }

                    //Log.i(TAG, "方位角:" + fAttitube[0]);
                    phone_slope = fAttitube[0];
                    //fAttitube[0] = 方位角
                    //fAttitube[1] = 前後の傾斜
                    //fAttitube[2] = 左右の傾斜
                }
            }
            @Override
            public void onAccuracyChanged(Sensor sensor, int i) {
            }

        };

        //iBeaconデータ取得ボタン
        final Button btnStart = (Button)this.findViewById(R.id.btnStart);
        btnStart.setOnClickListener(new View.OnClickListener () {
            @Override
            public void onClick(View v){
                if(flag_start == 0) {
                    flag_start = 1;
                    btnStart.setText("STOP");
                    Log.d(TAG, "STOP");
                }
                else{
                    flag_start = 0;
                    btnStart.setText("START");
                    Log.d(TAG, "START");
                }
            }
        });

        DataList = new ArrayAdapter<String>(this, android.R.layout.simple_list_item_1, data);
        listView.setAdapter(DataList);

    }


    @Override
    public void onBeaconServiceConnect() {
        beaconManager.setMonitorNotifier(new MonitorNotifier() {
            @Override
            public void didEnterRegion(Region region) {
                // 領域侵入時に実行
                Log.d(TAG, "didEnterRegion");
                //識別するビーコンのmajor minorを指定
                Identifier major = Identifier.parse("1");
                Identifier minor = Identifier.parse("4");

                try {
                    // レンジングの開始
                    beaconManager.startRangingBeaconsInRegion(new Region("unique-id-001", null, null, null));
                    Log.d(TAG, "1113");
                } catch(RemoteException e) {
                    // 例外が発生した場合
                    e.printStackTrace();
                }
            }

            @Override
            public void didExitRegion(Region region) {
                // 領域退出時に実行
                Log.d(TAG, "didExitRegion");
                try {
                    // レンジング停止
                    beaconManager.stopRangingBeaconsInRegion(new Region("unique-id-001", null, null, null));
                    Log.d(TAG, "1114");
                } catch(RemoteException e) {
                    // 例外が発生した場合
                    e.printStackTrace();
                }
            }

            @Override
            public void didDetermineStateForRegion(int i, Region region) {
                // 領域への侵入/退出のステータスが変化したときに実行
                Log.d(TAG, "didDetermineStateForRegion");
            }
        });


        beaconManager.setRangeNotifier(new RangeNotifier() {
            @Override
            public void didRangeBeaconsInRegion(Collection<Beacon> beacons, Region region) {

                String Beacon_data = "Beacon_data";
                int major;
                int minor;

                //補正したRSSI
                int correcton_rssi = 0;
                int index = 0;
                int i;
                int j;

                //スマホの角度
                int angle_phone;
                //ビーコンの角度
                int angle_beacon;

                //ビーコン1, 2, 3, 4の座標
                int[][] beacon_coordination = {{0, 0}, {0, 1}, {1, 1}, {1, 0}};


                //roomのx, y比
                double ratio_x = 180;
                double ratio_y = 180;
                //roomの角度
                int room_slope = 0;

                List<String> BeaconDatas = new ArrayList<String>();
                // 検出したビーコンの情報を全部Logに書き出す
                //Log.i(TAG, "aaaaaBBBB : " + flag_start);
                if (flag_start == 1) {
                    //Log.i(TAG, "aaaaas(´・ω・`)");
                    for (Beacon beacon : beacons) {
                        major = beacon.getId2().toInt();
                        minor = beacon.getId3().toInt();

                        //スマホの角度
                        angle_phone = 0;
                        //Beaconの角度
                        angle_beacon = 0;

                        //すべてのRSSI配列に値がはいっているか確認

                        all_rssi_flag = true;
                        for(i = 0; i < 4; i++){
                            if (rssi[i] == 0){
                                all_rssi_flag = false;
                                //Log.i(TAG, "すべてのRSSI配列に値がはいってるall_rssi_flag_for: " + all_rssi_flag );
                            }
                        }
                        // Log.i(TAG, "phone_ratio:all_rssi_flag_if:" + all_rssi_flag);

                        //4つのRSSIが配列に格納されていたらroom上のスマホの位置比を計算
                        i = beacon.getId3().toInt();
                        rssi[i -1] = beacon.getRssi();
                        if (major == 0 && all_rssi_flag) {
                            phonePosition(minor, beacon.getRssi());
                        }
                        Log.d(TAG, "listADD");
                    }
                }

                if (phone_ratio[0] !=0 && phone_ratio[1] != 0){
                    Beacon_data = "補正後x,y比:" + "(" + phone_ratio[0] + ", " + phone_ratio[1] +")";
                    Log.i(TAG, "(x, y)比 phone_ratio" + "(" + phone_ratio[0] + ", " + phone_ratio[1] +")!!!!");

                    final String final_Beacon_data = Beacon_data;

                    mHandler.post(new Runnable() {
                        @Override
                        public void run() {
                            //ListにBeacon_dataをData追加
                            DataList.add(final_Beacon_data);
                            Log.i(TAG, "ListADD_RSSI");
                        }
                    });
                }
            }
            });

        try {
            // モニタリングの開始
            beaconManager.startMonitoringBeaconsInRegion(new Region("unique-id-001", null, null, null));
        } catch(RemoteException e) {
            // 例外が発生した場合
            e.printStackTrace();
        }
    }

    @Override
    protected void onPause() {
        super.onPause();
        beaconManager.unbind(this);


        //リスナーの解除
        manager.unregisterListener(listener);

    }

    @Override
    protected void onResume() {
        super.onPause();
        beaconManager.bind(this);

        //スマホの方位角求める
        super.onResume();
        //リスナー設定:加速度
        manager.registerListener(
                listener,
                manager.getDefaultSensor(Sensor.TYPE_ACCELEROMETER),
                SensorManager.SENSOR_DELAY_FASTEST
        );

        //リスナー設定:地磁気
        manager.registerListener(
                listener,
                manager.getDefaultSensor(Sensor.TYPE_MAGNETIC_FIELD),
                SensorManager.SENSOR_DELAY_FASTEST
        );

    }

    public static void phonePosition (int minor, int original_rssi) {

        //補正したRSSI
        int correcton_rssi = 0;
        int index = 0;
        int i;
        int j;

        //スマホの角度
        int angle_phone;
        //ビーコンの角度
        int angle_beacon;

        //ビーコン1, 2, 3, 4の座標
        int[][] beacon_coordination = {{0, 0}, {0, 1}, {1, 1}, {1, 0}};

        //roomのx, y比
        double ratio_x = 180;
        double ratio_y = 180;
        //roomの角度
        int room_slope = 0;

        //スマホからBeaconまでの距離を返す関数
        distance[minor - 1] = beaconDistance(original_rssi);

        if (distance[0] != 0.0 && distance[1] != 0.0 && distance[2] != 0.0 && distance[3] != 0.0) {
            //補正前RSSIでスマホの位置座標をもとめる
            phone_ratio = phoneRatio(all_rssi_flag ? 1 : 0, beacon_coordination, distance, ratio_x, ratio_y);
            Log.i(TAG, "Beacon" + minor +" 補正前(x, y)比 phone_ratio" + "(" + phone_ratio[0] + ", " + phone_ratio[1] + ")");
            Log.i(TAG, "Beacon" + minor +" 補正前RSSI1 = " + rssi[0] + ", 2 = " + rssi[1] + ", 3 = " + rssi[2] + " 4 = " + rssi[3]);

            //ビーコンの傾きを求める
            angle_beacon = beaconAngle(minor, phone_ratio);
            //ビーコンの傾きを考慮してRSSI値を補正
            rssi[minor - 1] = beaconCorrection(angle_beacon, rssi[minor - 1]);
            Log.i(TAG, "Beacon" + minor + "傾き:" + angle_beacon + " ビーコン補正後RSSI 1 = " + rssi[0] + ", 2 = " + rssi[1] + ", 3 = " + rssi[2] + " 4 = " + rssi[3]);
            //スマホの傾きを求める
            angle_phone = phoneAngle(minor, phone_ratio, room_slope, phone_slope);
            //スマホの傾きを考慮してRSSI値を補正
            rssi[minor - 1] = phoneCorrection(angle_phone, rssi[minor - 1]);
            Log.i(TAG, "Beacon" + minor + "傾き:" + angle_phone + "スマホ補正後RSSI 1 = " + rssi[0] + ", 2 = " + rssi[1] + ", 3 = " + rssi[2] + " 4 = " + rssi[3]);

            //スマホからBeaconまでの距離を返す関数
            distance[minor - 1] = beaconDistance(rssi[minor - 1]);

            //補正後RSSIでスマホの位置座標をもとめる
            phone_ratio = phoneRatio(all_rssi_flag ? 1 : 0, beacon_coordination, distance, ratio_x, ratio_y);
            Log.i(TAG, "Beacon" + minor + " 補正後(x, y)比 phone_ratio" + "(" + phone_ratio[0] + ", " + phone_ratio[1] + ")\n");
        }
    }

    //スマホの角度とRSSIをなげると補正をかけたRSSIをかえす関数引数:(スマホ角度, RSSI) 戻り値:角度補正済みRSSI
    public static int phoneCorrection(int angle, int rssi){
        //引数の角度の補正値を格納する変数
        double correction_rssi = 1;
        //実験でもとめた30°ごとの補正値
        double[] correction_value = {1.0f, 1.074737f, 1.020823f, 1.012992f, 0.915333f, 0.94635f, 0.914798f, 1.010596f, 0.924614f, 1.05462f, 0.957603f, 0.973963f, 1.0f};

        correction_rssi = (angle - (30 * (angle / 30))) / 30 * (correction_value[(angle / 30) + 1] - correction_value[angle / 30]) + correction_value[angle / 30];
        //Log.i(TAG, "RSSIスマホ補正後:" + correction_rssi * rssi + " SSI:" + rssi + " angle:" + angle);

        return (int)(correction_rssi * rssi);
    }
    //ビーコンの角度とRSSIをなげると補正をかけたRSSIをかえす関数 引数:(ビーコン角度, RSSI) 戻り値:角度補正済みRSSI
    public static int beaconCorrection(int angle, int rssi){
        //引数の角度の補正値を格納する変数
        double correction_rssi = 1;
        //実験でもとめた30°ごとの補正値
        double[] correction_value = {1.0f, 1.106671753f, 1.1172685111f, 1.117849008f, 1.053278198f, 1.082484096f, 1.052625857f, 1.080279389f, 1.089982254f, 1.101893578f, 1.026139764f, 1.0155396f, 1.0f};

        correction_rssi = (angle - (30 * (angle / 30))) / 30 * (correction_value[(angle / 30) + 1] - correction_value[angle / 30]) + correction_value[angle / 30];
        //Log.i("RSSI補正値", "ビーコン補正後:" + correction_rssi * rssi + " RSSI:" + rssi + " angle:" + angle);

        return (int)(correction_rssi * rssi);
    }
    //ビーコンからのスマホまでの距離を返す関数 引数:補正済みRSSI 戻り値:ビーコンまでの距離
    public static  double beaconDistance(int correction_rssi){
        Log.i(TAG, "距離座標 RSSI:" + correction_rssi + "距離:" + (Math.exp((correction_rssi - -61.22) / -11.74)));
        return (Math.exp((correction_rssi - -61.22) / -11.74));
    }
    //渡された4つのRSSIの平均を求める関数 引数:データ数4つの一次元配列(RSSI値) 戻り値:4つのRSSIの平均
    public static double distanceWSum(double dictance[]){
        int i;
        double sum = 0;

        for(i = 0; i < 4; i++){
            sum += 1 / dictance[i];
        }
        return sum;
    }
    /*room内のスマホの座標を返す関数 
    引数:4つのRSSIが受信したか判定するフラグ ビーコン1, 2, 3, 4の座標 ビーコン1, 2, 3, 4からスマホへの距離 roomのx軸の幅 roomのy軸の幅
    戻り値:x, yの座標が格納された配列
     */
    public static  double[] phoneRatio(int all_rssi_flag, int[][] beacon_coordination, double[] distance, double ratio_x, double ratio_y){

        double[] ratio = {0, 0};
        int j;
        double distance_sum;

        //Log.i(TAG, "aaaphoneX_ratio" + all_rssi_flag+ " " + distance[0] + " " + distance[1]+ " " + distance[2]+ " " + distance[3]+ " " + ratio_x+ " " + ratio_y);

        if (all_rssi_flag == 1){
            // W = 1 / (4つのビーコンからの距離の和) 格納
            distance_sum = distanceWSum(distance);
            //room上でのx,y比を求める

            //Log.i(TAG,  "aaaphoneA_ratio[x] = " + ratio[0]);
            //Log.i(TAG,  "aaaphoneA_ratio[y] = " + ratio[1]);
            for (j = 0; j < 4; j++){
                ratio[0] += ((double)beacon_coordination[j][0] / distance[j])  * ratio_x;
                ratio[1] += ((double)beacon_coordination[j][1] / distance[j])  * ratio_y;
                Log.i(TAG,  "aaaphoneB_ratio[x] " + ratio[0] + "(x)" + " += " + (double)beacon_coordination[j][0] + " / " + (double)distance[j]);
                Log.i(TAG,  "aaaphoneB_ratio[y] " + ratio[1] + "(y)" + " += " + (double)beacon_coordination[j][1] + " / " + (double)distance[j]);
            }

            //Log.i(TAG, "aaaphoneC_ratio" + ratio[0] + ", " + ratio[1]);
        }

        return ratio;
    }
    /*現在のビーコンの向きからスマホと向い合せになるまでの角度を返す関数
    引数:ビーコンの番号 スマホの座標比
    戻り値:int型ビーコンの傾き(単位:度)
     */
    public static int beaconAngle(int beaconNum, double[] ratio){
        //roomのy軸を1とした時のxの倍率
        double room_ratio_x = ratio[0] / ratio[1];
        double slope = 0;

        if (beaconNum == 1){
            ratio[0] *= room_ratio_x;
            slope =  Math.atan2(ratio[0], ratio[1]);
        }
        else if (beaconNum == 2){
            ratio[0] *= room_ratio_x;
            slope =  Math.atan2((1 - ratio[1]), ratio[0]);
        }
        else if (beaconNum == 3){
            ratio[0] *= room_ratio_x;
            slope =  Math.atan2((1 - ratio[1]), (1 - ratio[0]));
        }
        else if (beaconNum == 4){
            ratio[0] *= room_ratio_x;
            slope =  Math.atan2(ratio[1], (1 - ratio[0]));

        }
        slope = slope / (Math.PI / 180);

        slope = slope % 360;
        if (slope < 0) {
            slope = slope + 360;
        }
        //Log.i(TAG, "beaconNum:" + beaconNum + "角度誤差:" + (int)slope);

        return (int)slope;
    }
    /*スマホが向いている方向からビーコンと向い合せになるまでの角度を返す関数
    引数:ビーコンの番号 スマホの座標比 roomの傾き スマホの傾き(北から今向いている方向までの角度)
    戻り値:int型スマホ正面から時計回りの角度(単位:度)
     */
    public static int phoneAngle(int beaconNum, double[] ratio, int room_slope, double phone_slope) {
        double room_ratio_x = ratio[0] / ratio[1];
        double slope = 0;

        if (beaconNum == 1){
            ratio[0] *= room_ratio_x;
            slope = (Math.atan2(ratio[1], ratio[0]) / (Math.PI / 180)) + room_slope - phone_slope + 180;
        }
        else if (beaconNum == 2){
            ratio[0] *= room_ratio_x;
            slope = (-1 * Math.atan2((1 - ratio[1]), ratio[0]) / (Math.PI / 180)) + room_slope - phone_slope;
        }
        else if (beaconNum == 3){
            ratio[0] *= room_ratio_x;
            slope = Math.atan2((1 - ratio[0]), (1 - ratio[1]) / (Math.PI / 180)) + room_slope - phone_slope;
        }
        else if (beaconNum == 4){
            ratio[0] *= room_ratio_x;
            slope = (-1 * Math.atan2(ratio[1], (1 - ratio[0]) / (Math.PI / 180)) + room_slope - phone_slope);

        }

        slope = slope % 360;
        if (slope < 0) {
            slope = slope + 360;
        }
        //Log.i(TAG, "beaconNum:" + beaconNum + "スマホ角度誤差:" + (int)slope);

        return (int)slope;

    }


    @Override
    public boolean onCreateOptionsMenu(Menu menu) {
        // Inflate the menu; this adds items to the action bar if it is present.
        getMenuInflater().inflate(R.menu.menu_main, menu);
        return true;
    }

    @Override
    public boolean onOptionsItemSelected(MenuItem item) {
        // Handle action bar item clicks here. The action bar will
        // automatically handle clicks on the Home/Up button, so long
        // as you specify a parent activity in AndroidManifest.xml.
        int id = item.getItemId();

        //noinspection SimplifiableIfStatement
        if(id == R.id.action_settings) {
            return true;
        }

        return super.onOptionsItemSelected(item);
    }

}

\end{verbatim}



\subsection{PictureActivity.java}
	\verbatiminput{./srcSwipeTalk/PictureActivity.java}
	\newpage

\subsection{activity\_document.xml}
	\verbatiminput{./srcSwipeTalk/activity_document.xml}
	\newpage
\subsection{activity\_image\_capture.xml}
	\verbatiminput{./srcSwipeTalk/activity_image_capture.xml}
	\newpage
\subsection{activity\_main.xml}
	\verbatiminput{./srcSwipeTalk/activity_main.xml}
	\newpage
\subsection{activity\_message.xml}
	\verbatiminput{./srcSwipeTalk/activity_message.xml}
	\newpage
\subsection{activity\_picture.xml}
	\verbatiminput{./srcSwipeTalk/activity_picture.xml}
	\newpage
\subsection{AndroidManifest.xml}
	\verbatiminput{./srcSwipeTalk/AndroidManifest.xml}
	\newpage
\subsection{list.xml}
	\verbatiminput{./srcSwipeTalk/list.xml}
	\newpage
\subsection{selectable\_grid\_item.xml}
	\verbatiminput{./srcSwipeTalk/selectable_grid_item.xml}
	\newpage
\subsection{titlebar.xml}
	\verbatiminput{./srcSwipeTalk/titlebar.xml}
	\newpage
\subsection{titlebar02.xml}
	\verbatiminput{./srcSwipeTalk/titlebar02.xml}
	\newpage



\section{SwipeTalkBase}
\subsection{BeaconCalibration.java}
	\verbatiminput{./srcSwipeTalkBase/BeaconCalibration.java}
	\newpage
\subsection{BLEFragment.java}
	\verbatiminput{./srcSwipeTalkBase/BLEFragment.java}
	\newpage
\subsection{ConvertSendableFormat.java}
	\verbatiminput{./srcSwipeTalkBase/ConvertSendableFormat.java}
	\newpage
\subsection{DeviceCoordinate.java}
	\verbatiminput{./srcSwipeTalkBase/DeviceCoordinate.java}
	\newpage
\subsection{IBeaconAdvertiseCallback.java}
	\verbatiminput{./srcSwipeTalkBase/IBeaconAdvertiseCallback.java}
	\newpage
\subsection{MainActivity.java}
	\verbatiminput{./srcSwipeTalkBase/MainActivity.java}
	\newpage
\subsection{SetBeaconActivity.java}
	\verbatiminput{./srcSwipeTalkBase/SetBeaconActivity.java}
	\newpage


\subsection{activity\_set\_beacon.xml}
	\verbatiminput{./srcSwipeTalkBase/activity_set_beacon.xml}
	\newpage
\subsection{AndroidManifest.xml}
	\verbatiminput{./srcSwipeTalkBase/AndroidManifest.xml}
	\newpage



\section{SwipeBomb}
\subsection{CameraPreview.java}
	\verbatiminput{./srcSwipeBomb/CameraPreview.java}
	\newpage
\subsection{MainActivity.java}
	\verbatiminput{./srcSwipeBomb/MainActivity.java}
	\newpage
\subsection{mSoundEffect.java}
	\verbatiminput{./srcSwipeBomb/mSoundEffect.java}
	\newpage

\subsection{activity\_main.xml}
	\verbatiminput{./srcSwipeBomb/activity_main.xml}
	\newpage
\subsection{AndroidManifest.xml}
	\verbatiminput{./srcSwipeBomb/AndroidManifest.xml}
	\newpage
\subsection{anim\_alpha0.xml}
	\verbatiminput{./srcSwipeBomb/anim_alpha0.xml}
	\newpage
\subsection{anim\_catch.xml}
	\verbatiminput{./srcSwipeBomb/anim_catch.xml}
	\newpage
\subsection{anim\_translate0.xml}
	\verbatiminput{./srcSwipeBomb/anim_translate0.xml}
	\newpage
\subsection{anim\_translate1.xml}
	\verbatiminput{./srcSwipeBomb/anim_translate1.xml}
	\newpage
\subsection{anim\_translate2.xml}
	\verbatiminput{./srcSwipeBomb/anim_translate2.xml}
	\newpage
\subsection{anim\_translate3.xml}
	\verbatiminput{./srcSwipeBomb/anim_translate3.xml}
	\newpage
\subsection{anim\_translate4.xml}
	\verbatiminput{./srcSwipeBomb/anim_translate4.xml}
	\newpage
\subsection{anim\_translate5.xml}
	\verbatiminput{./srcSwipeBomb/anim_translate5.xml}
	\newpage
\subsection{anim\_translate6.xml}
	\verbatiminput{./srcSwipeBomb/anim_translate6.xml}
	\newpage
\subsection{anim\_translate7.xml}
	\verbatiminput{./srcSwipeBomb/anim_translate7.xml}
	\newpage
\subsection{interpolator.xml}
	\verbatiminput{./srcSwipeBomb/interpolator.xml}
	\newpage
	




\end{document}