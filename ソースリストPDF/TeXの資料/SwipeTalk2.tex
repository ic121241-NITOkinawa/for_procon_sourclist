\documentclass[11pt,a4j]{jarticle}
\usepackage[dvipdfmx]{graphicx}
\usepackage{amsmath}
\usepackage{slashbox}
\usepackage{subfigure} 
\usepackage{here}
\usepackage[T1]{fontenc}
\usepackage{url}
\usepackage{verbatim}

\setlength{\textwidth}{170mm}
\setlength{\evensidemargin}{-5mm}
\setlength{\oddsidemargin}{-5mm}


\makeatletter
\newcommand{\figcaption}[1]{\def\@captype{figure}\caption{#1}}
\newcommand{\tblcaption}[1]{\def\@captype{table}\caption{#1}}
\makeatother


\begin{document}
\vspace*{-20mm}
{\Large

\begin{flushleft}
第2X回プログラミングコンテスト
\end{flushleft}

\vspace{-15mm}
\begin{flushright}
プログラムソースリスト
\end{flushright}
}

\vspace{20mm}

\begin{center}
{\huge

{自由部門 : 発表順番号(登録番号)  XX(20041)}

\vspace{20mm}
タイトル:「SwipeTalk」


\vspace{20mm}
学校名:沖縄工業高等専門学校\\

\hspace{0mm}
\vspace{0mm}

\begin{tabular}{r l l}
氏名:& 西原 & 希咲\\
	 & 辺土名 & 朝飛 \\
	 & 与座 & 章宙 \\
	 & 當間 & 環 \\
指導教員:& 正木 & 忠勝 \\
\end{tabular}
}

\end{center}


\thispagestyle{empty}
\newpage
\setcounter{page}{1}



\tableofcontents


\thispagestyle{empty}
\newpage
\setcounter{page}{1}




\section{SwipeTalk}
\subsection{DocumentActivity.java}
	\verbatiminput{./srcSwipeTalk/DocumentActivity.java}
	\newpage
\subsection{HueAdapter.java}
	\verbatiminput{./srcSwipeTalk/HueAdapter.java}
	\newpage	
\subsection{ImageCaptureActivity.java}
	\verbatiminput{./srcSwipeTalk/ImageCaptureActivity.java}
	\newpage
\subsection{MainActivity.java}
	\verbatiminput{./srcSwipeTalk/MainActivity.java}
	\newpage
\subsection{MessageActivity.java}
	\verbatiminput{./srcSwipeTalk/MessageActivity.java}
	\newpage
\subsection{MultiSelectableGridView.java}
	\verbatiminput{./srcSwipeTalk/MultiSelectableGridView.java}
	\newpage
\subsection{Phone\_PositionEstimation.java}
	\verbatiminput{./srcSwipeTalk/Phone_PositionEstimation.java}
	\newpage
\subsection{PictureActivity.java}
	\verbatiminput{./srcSwipeTalk/PictureActivity.java}
	\newpage


\subsection{activity\_document.xml}
	\verbatiminput{./srcSwipeTalk/activity_document.xml}
	\newpage
\subsection{activity\_image\_capture.xml}
	\verbatiminput{./srcSwipeTalk/activity_image_capture.xml}
	\newpage
\subsection{activity\_main.xml}
	\verbatiminput{./srcSwipeTalk/activity_main.xml}
	\newpage
\subsection{activity\_message.xml}
	\verbatiminput{./srcSwipeTalk/activity_message.xml}
	\newpage
\subsection{activity\_picture.xml}
	\verbatiminput{./srcSwipeTalk/activity_picture.xml}
	\newpage
\subsection{AndroidManifest.xml}
	\verbatiminput{./srcSwipeTalk/AndroidManifest.xml}
	\newpage
\subsection{list.xml}
	\verbatiminput{./srcSwipeTalk/list.xml}
	\newpage
\subsection{selectable\_grid\_item.xml}
	\verbatiminput{./srcSwipeTalk/selectable_grid_item.xml}
	\newpage
\subsection{titlebar.xml}
	\verbatiminput{./srcSwipeTalk/titlebar.xml}
	\newpage
\subsection{titlebar02.xml}
	\verbatiminput{./srcSwipeTalk/titlebar02.xml}
	\newpage



\section{SwipeTalkBase}
\subsection{BeaconCalibration.java}
	\verbatiminput{./srcSwipeTalkBase/BeaconCalibration.java}
	\newpage
\subsection{BLEFragment.java}
	\verbatiminput{./srcSwipeTalkBase/BLEFragment.java}
	\newpage
\subsection{ConvertSendableFormat.java}
	\verbatiminput{./srcSwipeTalkBase/ConvertSendableFormat.java}
	\newpage
\subsection{DeviceCoordinate.java}
	\verbatiminput{./srcSwipeTalkBase/DeviceCoordinate.java}
	\newpage
\subsection{IBeaconAdvertiseCallback.java}
	\verbatiminput{./srcSwipeTalkBase/IBeaconAdvertiseCallback.java}
	\newpage
\subsection{MainActivity.java}
	\verbatiminput{./srcSwipeTalkBase/MainActivity.java}
	\newpage
\subsection{SetBeaconActivity.java}
	\verbatiminput{./srcSwipeTalkBase/SetBeaconActivity.java}
	\newpage



\subsection{activity\_set\_beacon.xml}
	\verbatiminput{./srcSwipeTalkBase/activity_set_beacon.xml}
	\newpage
\subsection{AndroidManifest.xml}
	\verbatiminput{./srcSwipeTalkBase/AndroidManifest.xml}
	\newpage


\section{SwipeBomb}
\subsection{CameraPreview.java}
	\verbatiminput{./srcSwipeBomb/CameraPreview.java}
	\newpage
\subsection{MainActivity.java}
	\verbatiminput{./srcSwipeBomb/MainActivity.java}
	\newpage
\subsection{mSoundEffect.java}
	\verbatiminput{./srcSwipeBomb/mSoundEffect.java}
	\newpage
\subsection{aaaa}

	\begin{verbatim}
package com.ict.yzmock9_18;

import android.hardware.Camera;
import android.media.AudioManager;
import android.media.Image;
import android.media.MediaPlayer;
import android.media.SoundPool;
import android.os.CountDownTimer;
import android.support.v7.app.ActionBarActivity;
import android.os.Bundle;
import android.view.GestureDetector;
import android.view.Menu;
import android.view.MenuItem;
import android.view.MotionEvent;
import android.view.View;
import android.view.animation.Animation;
import android.view.animation.AnimationUtils;
import android.widget.FrameLayout;
import android.widget.ImageView;
import android.widget.TextView;
import android.widget.Toast;

public class MainActivity extends ActionBarActivity//implements Animation.AnimationListener
{
    //フリックの速度を格納する変数
    public float x_total = 0;
    public float y_total = 0;

    //ゲーム開始のためのフラグ
//    public boolean pinOff = false;
    //手榴弾を持っているフラグ
    public boolean grenadeHold = false;
    //リザルト画面のフラグ
    public boolean result = false;
    //フリック回数を数える
    public int flingCnt = 0;

    //フリックアクションを取得するためクラスを宣言
    private GestureDetector mGestureDetector;

    //カウントダウンタイマーを一秒刻みの10秒間に設定
    //するつもりだけど、ちょっとこいつ消しとく
    //final MyCountDownTimer bombTimer = new MyCountDownTimer(10000, 1000);

    //勝者の画面
    public void gameWon(View v){
        //リザルト画面フラグを立てる
        result = true;

        //リザルト画面にふさわしくない者共を非表示
        //リザルト画面に必要な者共を表示
        //xmlのビューを非表示にしてる
        findViewById(R.id.bomb).setVisibility(View.GONE);
        findViewById(R.id.num1).setVisibility(View.GONE);
        findViewById(R.id.btnX).setVisibility(View.GONE);
        findViewById(R.id.btnCnt1).setVisibility(View.GONE);
        findViewById(R.id.btnCnt2).setVisibility(View.GONE);
        findViewById(R.id.btnCnt3).setVisibility(View.GONE);
        findViewById(R.id.button).setVisibility(View.GONE);
        findViewById(R.id.button2).setVisibility(View.GONE);
        findViewById(R.id.button3).setVisibility(View.GONE);
        findViewById(R.id.numCnt).setVisibility(View.GONE);
        //xmlのビューを可視表示にしてる
        findViewById(R.id.stringWon).setVisibility(View.VISIBLE);
        findViewById(R.id.imageView2).setVisibility(View.VISIBLE);
        findViewById(R.id.imageView3).setVisibility(View.VISIBLE);

        //勝った時の音を再生
        mSoundEffect.playWon();
    }

    //敗者人の画面
    public void gameLose(View v) {
        //リザルト画面フラグを立てる
        result = true;

        //リザルト画面にふさわしくない者共を非表示
        //リザルト画面に必要な者共を表示
        //xmlのビューを非表示化
        findViewById(R.id.bomb).setVisibility(View.GONE);
        findViewById(R.id.num1).setVisibility(View.GONE);
        findViewById(R.id.btnX).setVisibility(View.GONE);
        findViewById(R.id.btnCnt1).setVisibility(View.GONE);
        findViewById(R.id.btnCnt2).setVisibility(View.GONE);
        findViewById(R.id.btnCnt3).setVisibility(View.GONE);
        findViewById(R.id.button).setVisibility(View.GONE);
        findViewById(R.id.button2).setVisibility(View.GONE);
        findViewById(R.id.button3).setVisibility(View.GONE);
        findViewById(R.id.numCnt).setVisibility(View.GONE);
        //xmlのビューを可視表示
        findViewById(R.id.stringLose).setVisibility(View.VISIBLE);
        findViewById(R.id.imageView2).setVisibility(View.VISIBLE);
        findViewById(R.id.imageView3).setVisibility(View.VISIBLE);

        //負けた時の音を再生
        mSoundEffect.playLose();
    }

    //ゲーム終了時にRETRYを選択した時、最初の画面に戻る
    public void retry(View v){
        //ゲーム画面に遷移するので、フラグを折る
        result = false;

        ImageView img = (ImageView)findViewById(R.id.bomb);
        img.setImageResource(R.drawable.grenade_pin);

        //ゲーム画面を読み込む
        //xmlのビュー非表示
        findViewById(R.id.num1).setVisibility(View.GONE);
        findViewById(R.id.num2).setVisibility(View.GONE);
        findViewById(R.id.num3).setVisibility(View.GONE);
        findViewById(R.id.stringLose).setVisibility(View.GONE);
        findViewById(R.id.stringWon).setVisibility(View.GONE);
        findViewById(R.id.imageView2).setVisibility(View.GONE);
        findViewById(R.id.imageView3).setVisibility(View.GONE);
        //xmlのビューを可視表示
        findViewById(R.id.bomb).setVisibility(View.VISIBLE);
        findViewById(R.id.btnX).setVisibility(View.VISIBLE);
        findViewById(R.id.btnCnt1).setVisibility(View.VISIBLE);
        findViewById(R.id.btnCnt2).setVisibility(View.VISIBLE);
        findViewById(R.id.btnCnt3).setVisibility(View.VISIBLE);
        findViewById(R.id.button).setVisibility(View.VISIBLE);
        findViewById(R.id.button2).setVisibility(View.VISIBLE);
        findViewById(R.id.button3).setVisibility(View.VISIBLE);
        findViewById(R.id.numCnt).setVisibility(View.VISIBLE);
    }

    //ゲーム開始時に右上のバツをおした時アプリ終了
    public void kill(View v){
        //アプリの終了
        MainActivity.this.finish();
    }

    //ゲーム終了画面でENDボタンをおした時アプリ終了
    public void end(View v){
        //アプリの終了
        MainActivity.this.finish();
    }

    //残り時間3秒に達した時、数字を表示する関数。(ではあるが今はボタンクリックに反応させてる)
    public void btnCnt1(View v){
        //1を表示、2と3を非表示
        findViewById(R.id.num1).setVisibility(View.VISIBLE);
        findViewById(R.id.num2).setVisibility(View.GONE);
        findViewById(R.id.num3).setVisibility(View.GONE);
    }
    public void btnCnt2(View v){
        //2を表示、1と3を非表示
        findViewById(R.id.num1).setVisibility(View.GONE);
        findViewById(R.id.num2).setVisibility(View.VISIBLE);
        findViewById(R.id.num3).setVisibility(View.GONE);
    }
    public void btnCnt3(View v){
        //3を表示、1と2を非表示
        findViewById(R.id.num1).setVisibility(View.GONE);
        findViewById(R.id.num2).setVisibility(View.GONE);
        findViewById(R.id.num3).setVisibility(View.VISIBLE);
    }

    //爆弾がきた時のアニメーションを設定
    public void grenadeCome(View v){
         //手榴弾を持っているフラグを立てる
        grenadeHold = true;
        
        //手榴弾が落ちた音、カランカラン……みたいなのを再生
        mSoundEffect.playCaught();

        //imgにxmlにあるIDが「bomb」の画像のポインタを代入
        ImageView img = (ImageView)findViewById(R.id.bomb);
        //xmlにあるIDが「bomb」に、画像フォルダの「grenade_cought」の画像のポインタを代入
        img.setImageResource(R.drawable.grenade_cought);//9/28加筆

        //アニメーションのクラスをgrenadeComeに代入
        Animation grenadeCome = AnimationUtils.loadAnimation(this, R.anim.anim_catch);
        
        //grenadeComeのアニメーションをimgに適用
        img.startAnimation(grenadeCome);
        //アニメーションが終わった時の見た目を画面に表示し続ける
        grenadeCome.setFillBefore(true);
        //バウンドさせる
        grenadeCome.setInterpolator(this, R.anim.interpolator);
    }

    //フリックされた方向をうけとって、それによって爆弾をアニメーションさせる
    public void bombThrow(double deg) {
        //爆弾持っているフラグを折る
        grenadeHold = false;

        //プレイ画面なら投げた時の音を再生させて、リザルト画面なら無音を再生
        if (!result) {
//            sp.play(mSoundThrew, 1.0F, 1.0F, 0, 0, 1.0F);
            mSoundEffect.playThrew();
        }else{
//            sp.play(mSoundSilent, 1.0F, 1.0F, 0, 0, 1.0F);
            mSoundEffect.playSilent();
        }
        
        //imgに、xmlにあるIDが「bomb」のポインタを代入
        ImageView img = (ImageView)findViewById(R.id.bomb);
        //キャッチして投げた時、爆弾の画像を斜めのままにするようにコメントアウト
        //img.setImageResource(R.drawable.grenade_pin_less);//9/28加筆
    
        //リザルト画面フラグが立っていたら、爆弾の画像を透明にする.
        if(result){
            //imgBombに,xmlにあるIDが「bomb」のポインタを代入
            //ImageView imgBomb = (ImageView)findViewById(R.id.bomb);
            //imgBombの画像を、「alpha1dot」に代入
            img.setImageResource(R.drawable.alpha1dot);
        }


        //リザルト画面では消す、プレイ画面なら再生させる
        if (!result){
            //スワイプの方向を8分割した方向、に入れるための変数
            int swipeWay;

            //アニメーションの定義
            //trans0~7に、アニメーションをセット
            //0左 1左上 2上 3右上 4右 5右下 6下 7左下 に飛んで行くアニメーション
            Animation trans0 = AnimationUtils.loadAnimation(this, R.anim.anim_translate0);
            Animation trans1 = AnimationUtils.loadAnimation(this, R.anim.anim_translate1);
            Animation trans2 = AnimationUtils.loadAnimation(this, R.anim.anim_translate2);
            Animation trans3 = AnimationUtils.loadAnimation(this, R.anim.anim_translate3);
            Animation trans4 = AnimationUtils.loadAnimation(this, R.anim.anim_translate4);
            Animation trans5 = AnimationUtils.loadAnimation(this, R.anim.anim_translate5);
            Animation trans6 = AnimationUtils.loadAnimation(this, R.anim.anim_translate6);
            Animation trans7 = AnimationUtils.loadAnimation(this, R.anim.anim_translate7);
            //Animation grenadeCome = AnimationUtils.loadAnimation(this, R.anim.anim_catch);
            //Animation cought = AnimationUtils.loadAnimation(this, R.anim.anim_catch);

            //フリック方向の判定
            /**swipeWayの数値とその方向(45度刻み)の対応
             * 左向   0
             * 左上   1
             * 上     2
             * 右上   3
             * 右     4
             * 右下   5
             * 下     6
             * 左下   7
             * 例外   0
            */
            //角度が、-180~180度まで与えられるので
            //それを8分割する
            if (deg >= -22.5 && deg < 22.5) {
                swipeWay = 0;
            } else if (deg >= 22.5 && deg < 67.5) {
                swipeWay = 1;
            } else if (deg >= 67.5 && deg < 112.5) {
                swipeWay = 2;
            } else if (deg >= 112.5 && deg < 157.5) {
                swipeWay = 3;
            } else if (   deg >= 157.5 && deg <= 180
                       || deg >= -180  && deg < -157.5){
                swipeWay = 4;
            } else if (deg >= -157.5 && deg < -112.5){
                swipeWay = 5;
            } else if (deg >= -112.5 && deg < -67.5){
                swipeWay = 6;
            } else if (deg >=  -67.5 && deg < -22.5){
                swipeWay = 7;
            }else{
                //一応例外処理
                swipeWay = 0;
            }

            //フリックされた方向によっていいアニメーションさせる
            switch (swipeWay){
                case 0:
                    //imgにtrans0のアニメーションを適用
                    img.startAnimation(trans0);
                    break;
                case 1:
                    img.startAnimation(trans1);
                    //imgにtrans0のアニメーションを適用
                    break;
                case 2:
                    img.startAnimation(trans2);
                    //imgにtrans0のアニメーションを適用
                    break;
                case 3:
                    img.startAnimation(trans3);
                    //imgにtrans0のアニメーションを適用
                    break;
                case 4:
                    img.startAnimation(trans4);
                    //imgにtrans0のアニメーションを適用
                    break;
                case 5:
                    img.startAnimation(trans5);
                    //imgにtrans0のアニメーションを適用
                    break;
                case 6:
                    img.startAnimation(trans6);
                    //imgにtrans0のアニメーションを適用
                    break;
                case 7:
                    img.startAnimation(trans7);
                    //imgにtrans0のアニメーションを適用
                    break;
                default:
                    //一応、例外処理
                    img.startAnimation(trans0);
                    //imgにtrans0のアニメーションを適用
            }
            //飛んでいったあとは、非表示
            img.setVisibility(View.GONE);
        }
    }

    // 画面の背景をカメラからの画像を表示するために使う
    private Camera mCam = null;
    private CameraPreview mCamPreview = null;



//    TextView timer = (TextView)findViewById(R.id.numCnt);
    TextView timer;// = (ImageView)findViewById(R.id.bomb);

    final MyCountDownTimer cdt = new MyCountDownTimer(10 * 1000, 1 * 1000);
    public class MyCountDownTimer extends CountDownTimer{

        public MyCountDownTimer(long millisInFuture, long countDownInterval) {
            super(millisInFuture, countDownInterval);
        }

        @Override
        public void onFinish() {
            // カウントダウン完了後に呼ばれる
            if(grenadeHold){
                //渡されるtxtCntに意味は無いが、引数にウィジェットを持たせないとダメ
                //最後に手榴弾持っていたら、負けた人の画面
                gameLose(timer);
            }else{
                //最後に手榴弾を持っていなかったら、勝者の画面
                gameWon(timer);
                timer.setText(Long.toString(millisInFuture / 1000));
            }
        }

        @Override
        public void onTick(long millisUntilFinished) {
            // インターバル(countDownInterval)毎に呼ばれる
//            timer.setText(Long.toString(millisUntilFinished/1000));
        }
    }

    @Override
    protected void onCreate(Bundle savedInstanceState) {
        super.onCreate(savedInstanceState);

        timer = (TextView)findViewById(R.id.numCnt);
        
        //activity_main.xmlを読み込んで画面に表示
        setContentView(R.layout.activity_main);

        // カメラインスタンスの取得
        try {
            mCam = Camera.open();
            //取得する画像は横に向いているので、90度回転して描画する
            mCam.setDisplayOrientation(90);
        } catch (Exception e) {
            // エラーを起こしたら、ゲーム終了(割りと危ない処理)
            this.finish();
        }

        // FrameLayout に CameraPreview クラスを設定
        //カメラの画像を取得する感じのやーつ
        FrameLayout preview = (FrameLayout)findViewById(R.id.cameraPreview);
        mCamPreview = new CameraPreview(this, mCam);
        preview.addView(mCamPreview);
        
        //全画面化する関数.XMLで定義したのでいらないけど、メモまでに書いている
        //this.setRequestedOrientation(ActivityInfo.SCREEN_ORIENTATION_PORTRAIT);

        mGestureDetector = new GestureDetector(this, mOnGestureListener);

        //カウントダウンタイマ
/*        new CountDownTimer(10 * 1000, 1 * 1000){
 *          //カウントダウンの数字を表示するためのもの
 *          //txtCntにxmlにあるIDが「cumCnt」のポインタを入れる
 *          TextView txtCnt = (TextView)findViewById(R.id.numCnt);
 *
 *          //1sのインターバルで呼ばれる
 *          public void onTick(long millisUntilFinished){
 *              //残り時間を秒単位で表示する
 *              txtCnt.setText(Long.toString(millisUntilFinished/1000));
 *          }
 *          public void onFinish(){
 *              //カウントダウン終了時、爆弾を持っていたら負けた人の画面
 *              //爆弾を持っていなかったら勝ちの画面
 *
 *              if(grenadeHold){
 *                  //渡されるtxtCntに意味は無いが、引数にウィジェットを持たせないとダメ
 *                  //最後に手榴弾持っていたら、負けた人の画面
 *                  gameLose(txtCnt);
 *              }else{
 *                  //最後に手榴弾を持っていなかったら、勝者の画面
 *                  gameWon(txtCnt);
 *                }
 *            }
 *        }.start();//onCreateにすぐStartを書いているので、アプリが起動したらカウントダウン開始
 *                  //フリックを開始したらスタートに変えたい(10/1追記)
 */

    }

    @Override
    protected void onResume(){
        super.onResume();
        //あらかじめ、効果音データを読み込んでおく
        new mSoundEffect(this);
    }

    @Override
    protected void onPause() {
        super.onPause();
        
        //アプリ停止時には読み込んだ音楽データを開放する
        mSoundEffect.setUnload();

        // カメラ破棄インスタンスを解放
        if (mCam != null) {
            mCam.release();
            mCam = null;
        }
    }

    @Override
    public boolean onCreateOptionsMenu(Menu menu) {
        // Inflate the menu; this adds items to the action bar if it is present.
        getMenuInflater().inflate(R.menu.menu_main, menu);
        return true;
    }

    @Override
    public boolean onOptionsItemSelected(MenuItem item) {
        // Handle action bar item clicks here. The action bar will
        // automatically handle clicks on the Home/Up button, so long
        // as you specify a parent activity in AndroidManifest.xml.
        int id = item.getItemId();

        //noinspection SimplifiableIfStatement
        if (id == R.id.action_settings) {
            return true;
        }

        return super.onOptionsItemSelected(item);
    }
    
    //タップがあれば呼び出される
    @Override
    public boolean onTouchEvent(MotionEvent event){
        //画面が触られたら、
        return mGestureDetector.onTouchEvent(event);
    }

    //フリックされたX方向とY方向の速度を取得する関数
    private final GestureDetector.SimpleOnGestureListener mOnGestureListener = new GestureDetector.SimpleOnGestureListener(){
        @Override
        public boolean onFling(MotionEvent e1, MotionEvent e2, float velocityX, float velocityY){
            if (flingCnt++ == 0)
            {
                cdt.start();
            }

            float xxx = x_total - velocityX;
            float yyy = y_total - velocityY;
            x_total = xxx;
            y_total = yyy;

            //そしてアニメーションする関数にぶん投げる
            //速度を取得して、そのベクトルを逆タンジェント関数に投げて角度にしてもらう
            //角度はラジアンで取得されるので、60分法に変換してもらう
            //正規の角度が貰えたら、その角度に応じてアニメーションする関数にぶん投げる
            bombThrow(Math.toDegrees(Math.atan2((double)yyy, (double)xxx)));

            x_total = 0;
            y_total = 0;

            return true;
        }
    };

    //アプリ終了時に呼ばれる
    @Override
    protected void onDestroy(){
        //アプリのメモリからの破棄あくして命令
        super.onDestroy();
    }
}

	\end{verbatim}

\subsection{activity\_main.xml}
	\verbatiminput{./srcSwipeBomb/activity_main.xml}
	\newpage
\subsection{AndroidManifest.xml}
	\verbatiminput{./srcSwipeBomb/AndroidManifest.xml}
	\newpage
\subsection{anim\_alpha0.xml}
	\verbatiminput{./srcSwipeBomb/anim_alpha0.xml}
	\newpage
\subsection{anim\_catch.xml}
	\verbatiminput{./srcSwipeBomb/anim_catch.xml}
	\newpage
\subsection{anim\_translate0.xml}
	\verbatiminput{./srcSwipeBomb/anim_translate0.xml}
	\newpage
\subsection{anim\_translate1.xml}
	\verbatiminput{./srcSwipeBomb/anim_translate1.xml}
	\newpage
\subsection{anim\_translate2.xml}
	\verbatiminput{./srcSwipeBomb/anim_translate2.xml}
	\newpage
\subsection{anim\_translate3.xml}
	\verbatiminput{./srcSwipeBomb/anim_translate3.xml}
	\newpage
\subsection{anim\_translate4.xml}
	\verbatiminput{./srcSwipeBomb/anim_translate4.xml}
	\newpage
\subsection{anim\_translate5.xml}
	\verbatiminput{./srcSwipeBomb/anim_translate5.xml}
	\newpage
\subsection{anim\_translate6.xml}
	\verbatiminput{./srcSwipeBomb/anim_translate6.xml}
	\newpage
\subsection{anim\_translate7.xml}
	\verbatiminput{./srcSwipeBomb/anim_translate7.xml}
	\newpage
\subsection{interpolator.xml}
	\verbatiminput{./srcSwipeBomb/interpolator.xml}
	\newpage
	




\end{document}