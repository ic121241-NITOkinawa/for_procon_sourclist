\documentclass[11pt,a4j]{jarticle}
\usepackage[dvipdfmx]{graphicx}
\usepackage{amsmath}
\usepackage{asmac}
\usepackage{slashbox}
\usepackage{subfigure} 
\usepackage{here}
\usepackage[T1]{fontenc}
\usepackage{url}
\usepackage{verbatim}
%\usepackage{seqsplit}
\usepackage{listings, jlisting}

\lstset{
    language=JAVA,%プログラミング言語によって変える。
    basicstyle={\normalsize \ttfamily},
%    keywordstyle={\color{OliveGreen}},
    %[2][3]はプログラミング言語によってあったり、なかったり
%    keywordstyle={[2]\color{colFunc}},
%    keywordstyle={[3]\color{CadetBlue}},%
%    commentstyle={\color{Brown}},
%    %identifierstyle={\color{colID}},
%    stringstyle=\color{blue},
    tabsize=2,
    frame=none,
    numbers=none,
    numberstyle={\ttfamily\small},
    breaklines=true,%折り返し
%    backgroundcolor={\color[gray]{.95}},
    captionpos=b
}

\setlength{\textwidth}{170mm}
\setlength{\evensidemargin}{-5mm}
\setlength{\oddsidemargin}{-5mm}


\makeatletter
\newcommand{\figcaption}[1]{\def\@captype{figure}\caption{#1}}
\newcommand{\tblcaption}[1]{\def\@captype{table}\caption{#1}}
\makeatother

\begin{document}
\vspace*{-20mm}
{\Large

\begin{flushleft}
第26回プログラミングコンテスト
\end{flushleft}

\vspace{-15mm}
\begin{flushright}
プログラムソースリスト
\end{flushright}
}

\vspace{20mm}

\begin{center}
{\huge

{自由部門 : 発表順番号(登録番号)  4(20041)}

\vspace{20mm}
タイトル:「SwipeTalk」


\vspace{20mm}
学校名:沖縄工業高等専門学校\\

\hspace{0mm}
\vspace{0mm}

\begin{tabular}{r l l}
氏名:& 西原 & 希咲\\
	 & 辺土名 & 朝飛 \\
	 & 与座 & 章宙 \\
	 & 當間 & 環 \\
指導教員:& 正木 & 忠勝 \\
\end{tabular}
}

\end{center}


\thispagestyle{empty}
\newpage

\thispagestyle{empty}
\tableofcontents

\thispagestyle{empty}
\newpage
\setcounter{page}{1}

\section{ソースリストの作成ご苦労さまです}
このPDFは, ソースリストの作成の手間をかなり削減させるために作りました.
この方法を使うためまずTeX環境をPCにインストールしてください.\\

\subsection{頭に入れてほしいこと3つ}
ソースリストを作るということは, 時間に追われている頃だと思いますので,
頭に入れてほしいことを三つだけを列挙します.
\begin{itemize}
\item 一つのプロジェクト毎に\verb|\|section\verb|{|Source\verb|.java}|する
\item 一つの貼り付けるソースファイル毎に\verb|\|subsection\verb|{|Source\verb|.cpp}|
\item 改ページのために\verb|\|subsection\verb|{|Source\verb|.xml}|毎に,\verb|\|newpage を挟む
\end{itemize}

だいたい, このPDFの元となったTeXを見ながらやればわかると思います.

\subsection{まず目次の作り方}
ソースリストを作るために,目次を作らなくてはいけません.TeXでは目次を自動で作ってくれます.\\
そのために,
\begin{verbatim}
\tableofcontents
\end{verbatim}
という命令を挿入します.こんな風に挿入してください.
\begin{verbatim}
\begin{document}
\thispagestyle{empty}
\tableofcontents

\thispagestyle{empty}
\newpage
\setcounter{page}{1}
\end{document}
\end{verbatim}

おめでとうございます.これで章(section)や節(subsection)を使うたびに目次が追加されていくようになりました.
次はソースファイルを参照する方法を見ていきましょう.

\newpage

\section{ソースコードをプロジェクトごとにまとめる.}


\section{SwipeTalk}
\subsection{DocumentActivity.java}
	\lstinputlisting{./srcSwipeTalk/DocumentActivity.java}
	\newpage
\subsection{HueAdapter.java}
	\lstinputlisting{./srcSwipeTalk/HueAdapter.java}
	\newpage	
\subsection{ImageCaptureActivity.java}
	\lstinputlisting{./srcSwipeTalk/ImageCaptureActivity.java}
	\newpage
\subsection{MainActivity.java}
	\lstinputlisting{./srcSwipeTalk/MainActivity.java}
	\newpage
\subsection{MessageActivity.java}
	\lstinputlisting{./srcSwipeTalk/MessageActivity.java}
	\newpage
\subsection{MultiSelectableGridView.java}
	\lstinputlisting{./srcSwipeTalk/MultiSelectableGridView.java}
	\newpage
\subsection{Phone\_PositionEstimation.java}
	\lstinputlisting{./srcSwipeTalk/Phone_PositionEstimation.java}
	\newpage
\subsection{PictureActivity.java}
	\lstinputlisting{./srcSwipeTalk/PictureActivity.java}
	\newpage

\subsection{activity\_document.xml}
	\lstinputlisting{./srcSwipeTalk/activity_document.xml}
	\newpage
\subsection{activity\_image\_capture.xml}
	\lstinputlisting{./srcSwipeTalk/activity_image_capture.xml}
	\newpage
\subsection{activity\_main.xml}
	\lstinputlisting{./srcSwipeTalk/activity_main.xml}
	\newpage
\subsection{activity\_message.xml}
	\lstinputlisting{./srcSwipeTalk/activity_message.xml}
	\newpage
\subsection{activity\_picture.xml}
	\lstinputlisting{./srcSwipeTalk/activity_picture.xml}
	\newpage
\subsection{AndroidManifest.xml}
	\lstinputlisting{./srcSwipeTalk/AndroidManifest.xml}
	\newpage
\subsection{list.xml}
	\lstinputlisting{./srcSwipeTalk/list.xml}
	\newpage
\subsection{selectable\_grid\_item.xml}
	\lstinputlisting{./srcSwipeTalk/selectable_grid_item.xml}
	\newpage
\subsection{titlebar.xml}
	\lstinputlisting{./srcSwipeTalk/titlebar.xml}
	\newpage
\subsection{titlebar02.xml}
	\lstinputlisting{./srcSwipeTalk/titlebar02.xml}
	\newpage



\section{SwipeTalkBase}
\subsection{BeaconCalibration.java}
	\lstinputlisting{./srcSwipeTalkBase/BeaconCalibration.java}
	\newpage
\subsection{BLEFragment.java}
	\lstinputlisting{./srcSwipeTalkBase/BLEFragment.java}
	\newpage
\subsection{ConvertSendableFormat.java}
	\lstinputlisting{./srcSwipeTalkBase/ConvertSendableFormat.java}
	\newpage
\subsection{DeviceCoordinate.java}
	\lstinputlisting{./srcSwipeTalkBase/DeviceCoordinate.java}
	\newpage
\subsection{IBeaconAdvertiseCallback.java}
	\lstinputlisting{./srcSwipeTalkBase/IBeaconAdvertiseCallback.java}
	\newpage
\subsection{MainActivity.java}
	\lstinputlisting{./srcSwipeTalkBase/MainActivity.java}
	\newpage
\subsection{SetBeaconActivity.java}
	\lstinputlisting{./srcSwipeTalkBase/SetBeaconActivity.java}
	\newpage


\subsection{activity\_set\_beacon.xml}
	\lstinputlisting{./srcSwipeTalkBase/activity_set_beacon.xml}
	\newpage
\subsection{AndroidManifest.xml}
	\lstinputlisting{./srcSwipeTalkBase/AndroidManifest.xml}
	\newpage



\section{SwipeBomb}
\subsection{CameraPreview.java}
	\lstinputlisting{./srcSwipeBomb/CameraPreview.java}
	\newpage
\subsection{MainActivity.java}
	\lstinputlisting{./srcSwipeBomb/MainActivity.java}
	\newpage
\subsection{mSoundEffect.java}
	\lstinputlisting{./srcSwipeBomb/mSoundEffect.java}
	\newpage

\subsection{activity\_main.xml}
	\lstinputlisting{./srcSwipeBomb/activity_main.xml}
	\newpage
\subsection{AndroidManifest.xml}
	\lstinputlisting{./srcSwipeBomb/AndroidManifest.xml}
	\newpage
\subsection{anim\_alpha0.xml}
	\lstinputlisting{./srcSwipeBomb/anim_alpha0.xml}
	\newpage
\subsection{anim\_catch.xml}
	\lstinputlisting{./srcSwipeBomb/anim_catch.xml}
	\newpage
\subsection{anim\_translate0.xml}
	\lstinputlisting{./srcSwipeBomb/anim_translate0.xml}
	\newpage
\subsection{anim\_translate1.xml}
	\lstinputlisting{./srcSwipeBomb/anim_translate1.xml}
	\newpage
\subsection{anim\_translate2.xml}
	\lstinputlisting{./srcSwipeBomb/anim_translate2.xml}
	\newpage
\subsection{anim\_translate3.xml}
	\lstinputlisting{./srcSwipeBomb/anim_translate3.xml}
	\newpage
\subsection{anim\_translate4.xml}
	\lstinputlisting{./srcSwipeBomb/anim_translate4.xml}
	\newpage
\subsection{anim\_translate5.xml}
	\lstinputlisting{./srcSwipeBomb/anim_translate5.xml}
	\newpage
\subsection{anim\_translate6.xml}
	\lstinputlisting{./srcSwipeBomb/anim_translate6.xml}
	\newpage
\subsection{anim\_translate7.xml}
	\lstinputlisting{./srcSwipeBomb/anim_translate7.xml}
	\newpage
\subsection{interpolator.xml}
	\lstinputlisting{./srcSwipeBomb/interpolator.xml}
	\newpage
	

\end{document}